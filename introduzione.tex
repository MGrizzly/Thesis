\chapter*{Introduzione}                 %crea l'introduzione (un capitolo
                                       %   non numerato)
%%%%%%%%%%%%%%%%%%%%%%%%%%%%%%%%%%%%%%%%%imposta l'intestazione di pagina
\rhead[\fancyplain{}{\bfseries
INTRODUZIONE}]{\fancyplain{}{\bfseries\thepage}}
\lhead[\fancyplain{}{\bfseries\thepage}]{\fancyplain{}{\bfseries
INTRODUZIONE}}
%%%%%%%%%%%%%%%%%%%%%%%%%%%%%%%%%%%%%%%%%aggiunge la voce Introduzione
                                        %   nell'indice
\addcontentsline{toc}{chapter}{Introduzione}

Le Security Token Offerings, abbreviate in STO, sono un fenomeno recente che si è diffuso a partire dalla seconda metà del 2017 mantenendo inizialmente la connotazione di Initial Coin Offerings con una maggiore attenzione per la regolamentazione e solo più tardi differenziate nella loro connotazione attuale di token che dal punto di vista giuridico tendono a configurarsi come securities, al pari di altri strumenti finanziari.  

Lo scopo di questo lavoro di tesi è stato approfondire la comprensione di questo fenomeno in particolare analizzandone le motivazioni, le caratteristiche e metodologie con le quali queste STO vengono realizzate.

In primo luogo, si definiranno le caratteristiche delle Initial Coin Offerings e successivamente verranno individuati i criteri per identificare correttamente la differenza tra queste e le Security Token Offerings. Una volta posti questi criteri si procederà ad analizzare i dati delle Security Token Offerings individuate al fine di poter descrivere le metriche di questo fenomeno. Una volta comparate le principali implementazioni di questo modello verrà analizzata la rilevanza delle Security Token Offerings e le prospettive di diffusione del fenomeno. 



%%%%%%%%%%%%%%%%%%%%%%%%%%%%%%%%%%%%%%%%%non numera l'ultima pagina sinistra
\clearpage{\pagestyle{empty}\cleardoublepage}