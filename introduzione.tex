\chapter*{Introduzione}                 %crea l'introduzione (un capitolo
                                       %   non numerato)
%%%%%%%%%%%%%%%%%%%%%%%%%%%%%%%%%%%%%%%%%imposta l'intestazione di pagina
\rhead[\fancyplain{}{\bfseries
INTRODUZIONE}]{\fancyplain{}{\bfseries\thepage}}
\lhead[\fancyplain{}{\bfseries\thepage}]{\fancyplain{}{\bfseries
INTRODUZIONE}}
%%%%%%%%%%%%%%%%%%%%%%%%%%%%%%%%%%%%%%%%%aggiunge la voce Introduzione
                                        %   nell'indice
\addcontentsline{toc}{chapter}{Introduzione}

++AGGIUNGERE FUNZIONAMENTO BITCOIN?

Le security token offerings, abbreviate in STO, sono un fenomeno recente che si è diffuso a partire dalla seconda metà del 2017 mantenendo inizialmente la connotazione di initial coin offerings debolmente regolamentate e solo più tardi differenziate nella loro connotazione attuale di token che dal punto di vista giuridico tendono a configurarsi come securities, al pari di altri strumenti finanziari.  

Lo scopo di questo lavoro di tesi è stato approfondire la comprensione di questo fenomeno in particolare analizzandone le caratteristiche, le cause e metodologie con le quali queste STO vengono realizzate.

In primo luogo, si definiranno le caratteristiche delle initial coin offerings e successivamente verranno individuati i criteri per identificare correttamente la differenza tra queste e le security token offerings. Una volta posti questi criteri si procederà ad analizzare i dati delle security token offerings individuate al fine di poter descrivere le metriche di questo fenomeno. Una volta valutata la rilevanza delle security token offering e le prospettive di crescita e diffusione verranno analizzate e comparate le principali implementazioni di questo modello.  



%%%%%%%%%%%%%%%%%%%%%%%%%%%%%%%%%%%%%%%%%non numera l'ultima pagina sinistra
\clearpage{\pagestyle{empty}\cleardoublepage}