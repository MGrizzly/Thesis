\chapter*{Introduzione}                 %crea l'introduzione (un capitolo
                                       %   non numerato)
%%%%%%%%%%%%%%%%%%%%%%%%%%%%%%%%%%%%%%%%%imposta l'intestazione di pagina
\rhead[\fancyplain{}{\bfseries
INTRODUZIONE}]{\fancyplain{}{\bfseries\thepage}}
\lhead[\fancyplain{}{\bfseries\thepage}]{\fancyplain{}{\bfseries
INTRODUZIONE}}
%%%%%%%%%%%%%%%%%%%%%%%%%%%%%%%%%%%%%%%%%aggiunge la voce Introduzione
                                        %   nell'indice
\addcontentsline{toc}{chapter}{Introduzione}

Le \textit{Security Token Offerings}, abbreviate in \textit{STOs}, sono un fenomeno recente che si è diffuso a partire dalla seconda metà del 2017 mantenendo inizialmente la connotazione di \textit{Initial Coin Offerings (ICOs)}, per poi prestare maggiore attenzione alla regolamentazione e differenziarsi in \textit{token sales} in cui il token è uno strumento finanziario regolamentato. 

Come si avrà modo di osservare, questo cambio di paradigma è ciò che contraddistingue le STOs. Nel 2017 le ICOs hanno raggiunto un picco di popolarità per poi la maggior parte fallire in meno di un anno, facendo capire agli investitori che le ICOs sono state una bolla speculativa. Nonostante ciò, la validità del modello di raccolta di capitale tramite la vendita di token basati su tecnologia Blockchain non è stata messa in discussione. Proprio per questo sono nate le STOs, delle token sales in cui il token è uno strumento finanziario, che offre tutela agli investitori. 

Lo scopo di questo lavoro di tesi è stato approfondire la comprensione di questo fenomeno in particolare analizzandone le motivazioni, le caratteristiche e le metodologie con le quali queste STOs vengono realizzate.

La tesi è strutturata come segue:
\begin{itemize}
    \item Nel primo capitolo si ripercorre la storia delle monete digitali per capire le complessità del processo di diffusione di queste. Vengono analizzate nel dettaglio le fasi della creazione e diffusione sia di Bitcoin che di Ethereum. Una volta comprese le motivazioni che hanno portato alla nascita di queste monete ne verrà considerata un'applicazione particolare, la raccolta di capitale. Di conseguenza, vengono confrontate le caratteristiche di diverse tipologie di token, in modo da poterli identificare e distinguere tra loro. 
    
    \item Nel secondo capitolo vengono presentati i protocolli principali con i quali le STOs vengono realizzate. Tra questi vengono prese in considerazione tre categorie principali. La prima categoria comprende lo standard ERC-20 e altri standard proprietari che estendono ERC-20. Nella seconda categoria vengono presentati gli standard dedicati a beni non fungibili, ovvero lo standard ERC-721 e il suo successore, lo standard ERC-1155. Infine, l'ultima categoria è lo standard ERC-1400 che comprende altri standard per i security tokens. 
    
    \item Nel terzo capitolo si discute dei principali driver di innovazione nel campo delle STOs. In primo luogo vengono presentate le IEOs, dei particolari tipi di token sales che avvengono su exchange dedicati.  Successivamente, si discute del ruolo di Libra, la criptovaluta proposta da Facebook, quale apripista per aprire un dialogo con i legislatori e stabilire un framework normativo chiaro. 
    
    \item Nel quarto capitolo vengono presentati i dati 
    ottenuti dopo una difficile ricerca. La difficoltà riscontrata è dovuta alla natura stessa delle STOs: le STOs si rivolgono esclusivamente ad investitori autorizzati, non ai consumatori. I dati presentati vengono poi analizzati per poter elaborare alcune considerazioni riguardo alle STOs. 
\end{itemize}

%%%%%%%%%%%%%%%%%%%%%%%%%%%%%%%%%%%%%%%%%non numera l'ultima pagina sinistra
\clearpage{\pagestyle{empty}\cleardoublepage}