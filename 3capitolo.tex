\chapter{Sviluppi futuri delle STOs}
\section{Initial Exchange Offerings (IEOs)}
Le Initial Exchnge Offerings, abbreviate in IEOs, sono delle sto particolari poiché la piattaforma utilizzata per emettere il token è fornita da un exchange. Quest'ultimo quindi si occupa di emettere il token e di listarlo, permettendo il trading secondario, spesso con diritti di esclusività. Questa integrazione favorisce notevolmente l'azienda che voglia emettere un security token. Infatti integrare la fase di issuing e di trading in un exchange permette di facilitare le procedure di compliance quali ad esempio le procedure di Know Your Customer e di Anti Money 
Laundering poiché già previste dall'exchange. Questa soluzione rappresenta il modo più semplice per emettere un security token, nonché in media la più costosa, in base all'exchange a cui ci si affida. Un altro svantaggio delle IEO è la limitazione dei potenziali investitori raggiungibili. 
Nonostante ciò, il fenomeno delle IEO risulta rilevante poiché appetibile ad un amplio share di aziende in quanto non richiede un grande sforzo organizzativo e tecnico. 
Ad oggi gli exchange più grandi che offrono questa possibilità sono: 
-binance 
-etc..
Altri exchange  invece stanno realizzando piattaforme per offrire IEO, tra queste troviamo Coinbase, etc..
L'interesse mostrato sia dagli exchange che dalle aziende interessate a creare un security token risulta chiaro ma per capire le dimensioni di questo fenomeno ne verranno analizzati alcuni dati quantitativi nel prossimo capitolo. 
\section{Libra}
Libra è una criptovaluta e un sistema di pagamento proposto da Facebook nel 2019. Sebbene sia stato dibattuto ampiamente il valore tecnico di libra, la rilevanza di libra è dovuta alla sfida regolatoria che pone agli stati. Il timore ben fondato di quest'ultimi è che la criptovaluta abbia una rilevanza sproporzionata a suo favore rispetto alle valute nazionali. A differenza di quanto accaduto con Bitcoin e con altre criptovalute, Libra è il primo caso in cui un'azienda privata di tali dimensioni mira ad emettere un token utilizzabile in un contesto internazionale così amplio. Ciò impone agli stati una sfida legislativa non banale che li sta forzando a legiferare in modo chiaro sugli ambiti di utilizzo delle criptovalute. In questo senso Libra sta svolgendo il ruolo di apripista laddove altre criptovaluta hanno fallito a causa di scarsa rilevanza o a causa di una scarsa recettività del legislatore. 
Il lancio di Libra è previsto per il 2020 ma sono già stati espressi commenti riguardo a possibili ritardi, dovuti al fatto che Facebook voglia operare in un ambito di chiarezza legislativa, con particolare riguardo per l'ambito statunitense. 