\chapter{Sviluppi futuri delle STOs}
\section{Initial Exchange Offerings (IEOs)}
Le \textit{Initial Exchange Offerings}, abbreviate in \textit{IEOs}, furono inizialmente introdotte nel 2017 ma non si sono affermate prima dell'inizio di quest'anno. Le IEOs consistono in STOs condotte attraverso un exchange. L'exchange quindi si occupa di emettere il token e di listarlo, permettendo il trading secondario, spesso con diritti di esclusività. All'exchange spettano inoltre gli obblighi di gestione del token. 

Potenzialmente la partecipazione degli investitori non è limitata ma di fatto viene circoscritta alla base di utenti che utilizza l'exchange. Questo avviene poiché gli accordi tra un azienda ed un exchange per l'emissione di un token di norma contengono clausole di esclusività. 
La relazione dell'issuer con l'exchange presenta anche dei vantaggi dal punto di vista dell'investitore. Oltre alle garanzie legali tipiche degli strumenti finanziari, le IEOs forniscono maggiore fiducia in quanto è nell'interesse dell'exchange garantire la qualità della IEO e del progetto per cui vengono realizzate. Ciò sprona l'exchange ad eseguire approfondimenti ulteriori sulla qualità dell'investimento. Un altro fattore a favore delle IEOs è la possibilità di utilizzare un infrastruttura preesistente con standard di sicurezza maggiori, un servizio di \textit{customer care}, meccanismi di \textit{KYC/AML} già sviluppati e infine la possibilità di avere una liquidità al momento del lancio della token sale. 
Queste caratteristiche permettono all'azienda che voglia emettere il token di delegare quasi completamente il processo di emissione dei tokens e di fare leva sulla piattaforma offerta dall'exchange per i contatti con gli investitori. Allo stesso tempo, l'investitore ha il vantaggio di non dover ripetere le procedure di KYC/AML per ogni STO in cui vuole investire e di poter scambiare il token con liquidità elevata appena cessata la token sale.

Ad oggi c'è molto interesse per le IEOs, soprattutto da parte degli exchanges. Tra le piattaforme per IEOs più famose possiamo citare:
\begin{itemize}
    \item Binance Launchpad
    \item Huobi Prime
    \item OKEx Jumpstart
    \item Coineal Launchpad 
    \item Bittrex
    \item Kucoin Spotlight
    \item Shortex 
    \item Probit Launchpad 
    \item Coinbene MoonBase
    \item Bitfinex Token Sales
    \item Coinbase (non annunciato ufficialmente) 
\end{itemize}
L'interesse mostrato sia dagli exchanges che dalle aziende interessate a creare un security token risulta chiaro ma per capire le dimensioni di questo fenomeno ne verranno presentati alcuni dati quantitativi nel prossimo capitolo.

Ad oggi non esistono standard specifici per le IEOs: è a discrezione dell'exchange valutare quale degli standard già presenti si adatti meglio alla sua infrastruttura. Solitamente un exchange ha già la necessità di supportare gli standard utilizzati nelle STOs ed è quindi ragionevole pensare che continui ad essere adottato uno degli standard già disponibili. 

\section{Libra}
Libra è una criptovaluta e un sistema di pagamento proposto da Facebook nel 2019. Al momento la data di lancio del progetto è prevista per il 2020 ma sono già stati espressi commenti riguardo a possibili ritardi. 
A livello tecnico libra è una criptovaluta a cui corrisponde un collaterale composto da un portafoglio di valute al fine di limitarne la volatilità. La gestione di Libra sarà assegnata ad un consorzio, \textit{Libra Association}, che per ora conta 27 membri. Ogni membro verserà 10 milioni di dollari per poter garantire il valore della criptovaluta. 
Libra è il primo caso in cui un'azienda privata di tali dimensioni mira ad emettere un token utilizzabile in un contesto internazionale così amplio. In questo senso Libra sta svolgendo il ruolo di apripista laddove altre criptovaluta hanno fallito a causa di scarsa rilevanza o a causa di uno scarso interesse del legislatore. Citando le parole di Mark Zuckerberg come riportate da \textit{The Verge}: 

\textit{‘‘.. And we have this bigger, or at least more exotic, project around Libra, which is to try to stand up a new kind of digital money that can work globally, [and] that will be stable … But it’s a big idea, and it’s a new type of system, especially to be implemented by big companies. We’re not the only ones doing this. We’ve led that the thinking and development on it so far, but the idea is to do this as an independent association, which is what we announced with about 27 other companies. By the time it launches, we expect we’ll have 100 or more companies as part of it. But part of what we’re trying to do overall on these big projects now that touch very socially important aspects of society is have a more consultative approach. So not just show up and say, “Alright, here we’re launching this. here’s a product, your app got updated, now you can start buying Libras and sending them around.” We want to make sure. We get that there are real issues. Finance is a very heavily regulated space. There’s a lot of important issues that need to be dealt with in preventing money laundering, preventing financing of terrorists and people who the different governments say you can’t do business with. There are a lot of requirements on knowing who your customers are. We already focus a lot on real identity, across especially Facebook, so there’s even more that we need to do in order to have this kind of a product. And we’re committed to doing that well, and part of doing that well is not just building the internal tools and showing up and saying, “Hey, we think we’ve solved this,” but addressing and meeting with all the regulators up front, hearing their concerns, hearing what they think we should be doing, making sure other folks in the consortium are handling this appropriately. ..’’}

Risulta quindi chiaro come il focus dei creatori di Libra non sia l'aspetto tecnologico del progetto, bensì quello di avviare un dialogo con i legislatori.  Questo aspetto rappresenta il più importante contributo innovativo apportato da Libra. Il progetto ha già ricevuto critiche ed una forte opposizione dalle banche centrali, da membri del governo statunitense e da parte del governo francese. Il ministro della finanza francese Bruno Le Maire ha dichiarato durante il \textit{G7} che la Francia non permetterà lo sviluppo di Libra poiché rappresenta una minaccia alla sovranità monetaria delle nazioni europee. Pareri altrettanto preoccupanti sono stati espressi dagli altri ministri riuniti al \textit{G7}. Risulta palese che l'introduzione di Libra andrà a fare chiarezza sullo stato delle regolamentazioni anche a livello internazionale. Questo passaggio è probabilmente il fattore fondamentale per determinare le possibilità di espansione nei prossimi anni del mercato degli strumenti finanziari basati su tecnologia blockchain. 

