\begin{thebibliography}{90}             %crea l'ambiente bibliografia
\rhead[\fancyplain{}{\bfseries \leftmark}]{\fancyplain{}{\bfseries
\thepage}}
%%%%%%%%%%%%%%%%%%%%%%%%%%%%%%%%%%%%%%%%%aggiunge la voce Bibliografia
                                        %   nell'indice
\addcontentsline{toc}{chapter}{Bibliografia}
%%%%%%%%%%%%%%%%%%%%%%%%%%%%%%%%%%%%%%%%%provare anche questo comando:
%%%%%%%%%%%\addcontentsline{toc}{chapter}{\numberline{}{Bibliografia}}
\bibitem{K1} S. Nakamoto, “Bitcoin: A Peer-to-Peer Electronic Cash System,” p. 9.
\bibitem{K2} Merkle, R. C. (1978). Secure communications over insecure channels. Communications of the ACM, 21(4), 294-299.
\bibitem{K3} Diffie, W., and Hellman, M. New directions in cryptography. IEEE Trans. on Inform. IT-22, 6 (Nov. 1976), 644-654.
\bibitem{K4} Merkle, R. C. (1987, August). A digital signature based on a conventional encryption function. In Conference on the theory and application of cryptographic techniques (pp. 369-378). Springer, Berlin, Heidelberg.
\bibitem{K5} Bayer, D., Haber, S., \& Stornetta, W. S. (1993). Improving the efficiency and reliability of digital time-stamping. In Sequences Ii (pp. 329-334). Springer, New York, NY.
\bibitem{K6} Chaum, D. (1983). Blind signatures for untraceable payments. In Advances in cryptology (pp. 199-203). Springer, Boston, MA.
\bibitem{K7} Chaum, D., Fiat, A., \& Naor, M. (1988, August). Untraceable electronic cash. In Conference on the Theory and Application of Cryptography (pp. 319-327). Springer, New York, NY.
\bibitem{K8} Schoenmakers, B. (1998). Security Aspects of the EcashTM Payment System. In State of the Art in Applied Cryptography: Course on Computer Security and Industrial Cryptography Leuven, Belgium, June 3–6, 1997 Revised Lectures (pp. 338–352). Berlin, Heidelberg: Springer Berlin Heidelberg. https://doi.org/10.1007/3-540-49248-8\_16
\bibitem{K9} Back, A. (2002). Hashcash-a denial of service counter-measure.
\bibitem{K10} Dwork, C., \& Naor, M. (1992, August). Pricing via processing or combatting junk mail. In Annual International Cryptology Conference (pp. 139-147). Springer, Berlin, Heidelberg.
\bibitem{K11} Jakobsson, M., \& Juels, A. (1999). Proofs of work and bread pudding protocols. In Secure Information Networks (pp. 258-272). Springer, Boston, MA.
\bibitem{K12} Dai, W. (1998) B-Money. http://www.weidai.com/bmoney.txt
\bibitem{K13} Finney, H. (2004). Reusable proofs of work (rpow).
\bibitem{K14} Szabo, N. (2008). Bit gold.
\bibitem{K15} Grigg, I. (2017). On the intersection of Ricardian and Smart Contracts.
\bibitem{K16} Chohan, U. W. (2017). A history of bitcoin.
\bibitem{K17} Network, F. C. E. (2013). Application of FinCEN’s regulations to persons administering, exchanging, or using virtual currencies. United States Department of the Treasury.
\bibitem{K18} De Filippi, P. (2014). Bitcoin: a regulatory nightmare to a libertarian dream. Internet Policy Review, 3(2).
\bibitem{K19} Ponsford, M. P. (2015). A comparative analysis of Bitcoin and other decentralised virtual currencies: legal regulation in the people's republic of China, Canada, and the United States. HKJ Legal Stud., 9, 29.
\bibitem{K20} Buterin, V. (2013). Ethereum white paper. GitHub repository, 22-23.
\bibitem{K21} Buterin V. A Prehistory of the Ethereum Protocol, https://vitalik.ca/general/2017/09/14/prehistory.html
\bibitem{K22} Wood, G. (2014). Ethereum: A secure decentralised generalised transaction ledger. Ethereum project yellow paper, 151(2014), 1-32.
\bibitem{K23} Willett, J. R., Hidskes, M., Johnston, D., Gross, R., \& Schneider, M. (2016). Omni Protocol Specification (formerly Mastercoin). white paper), accessed January, 28.
\bibitem{K24} CoinMarketCap, (2018). Cryptocurrency market capitalizations. Retrieved on September, 25, 2019.
\bibitem{K25} Jelurida, (2013), NextCoin, https://www.jelurida.com/nxt
\bibitem{K26} Counterparty, (2014), Counterparty. https://counterparty.io
\bibitem{K27}  MaidSafe (2014), MaidsafeCoin. https://maidsafe.net/
\bibitem{K28} SwarmCity (2014), Swarm. https://swarm.city/
\bibitem{K29} Peterson, J., \& Krug, J. (2015). Augur: a decentralized, open-source platform for prediction markets. arXiv preprint arXiv:1501.01042.
\bibitem{K30} Mazzorana-Kremer, F. (2019). Blockchain-Based Equity and STOs: Towards a Liquid Market for SME Financing?. Theoretical Economics Letters, 9(5), 1534-1552.
\bibitem{K31}  Loss, L. (1983). Fundamentals of securities regulation. Aspen Publishers Online.
\bibitem{K32} Clayton, J. (2017). Statement on cryptocurrencies and initial coin offerings. world.
\bibitem{K33} Hughes, S. D. (2017). Cryptocurrency Regulations and Enforcement in the US. W. St. UL Rev., 45, 1.
\bibitem{K34} Lgs, D. (24). febbraio 1998, n. 58. Testo unico delle disposizioni in materia di intermediazione finanziaria, ai sensi degli articoli 8 e 21 della legge 6 febbraio 1996, (52), 9.
\bibitem{K35} Ante, L., \& Fiedler, I. (2019). Cheap Signals in Security Token Offerings.
\bibitem{K36} Hayes, A. (2015). What factors give cryptocurrencies their value: An empirical analysis. Available at SSRN 2579445.
\bibitem{K37} Vogelsteller, F., & Buterin, V. (2015). Erc-20 token standard. Ethereum Foundation (Stiftung Ethereum), Zug, Switzerland.
\bibitem{K38}
\bibitem{K39}
\bibitem{K40}
\bibitem{K41}
\bibitem{K42}
\bibitem{K43}
\bibitem{K44}
\bibitem{K45}
\bibitem{K46}
\bibitem{K47}
\bibitem{K48}
\bibitem{K49}
\bibitem{K50}
\end{thebibliography}